\documentclass[xcolor=table,aspectratio=169]{beamer}
\usetheme{Madrid}
\usepackage{adjustbox}
\usepackage{dcolumn}
\newcolumntype{d}[1]{D{.}{.}{#1}}
%\usetheme{metropolis}
\usepackage[style=verbose-note, sorting=none, sortcites=true, maxnames=1, giveninits=true, autocite=superscript, doi=false, url=false, isbn=false, backend=biber, citetracker=false, pagetracker=false, bibencoding=utf8, eprint=false]{biblatex}
% \usepackage[backend=bibtex,style=authoryear-comp,citestyle=authoryear-comp,firstinits=true,sorting=none,maxnames=1,doi=false,isbn=false,url=false,eprint=false]{biblatex}
\usepackage[T1]{fontenc}
\usepackage[normalem]{ulem}

\definecolor{twitter_blue}{HTML}{1da1f2}
\input{seaborn_colours.tex}

% Gobbling first names

\AtEveryCitekey{%
   \clearfield{shorttitle}%
   \clearfield{month}%
   \clearfield{day}%
   \ifentrytype{article}{%
      \clearfield{title}%
   }{}
   }
\ExecuteBibliographyOptions[online]{eprint=true}

% "blindfootcite" is the equivalent of "footcite" except the number marker does not appear
\newcommand\blfootcite[1]{%
  \begingroup
  \renewcommand\thefootnote{}\footnote{\hspace{-4ex}\cite{#1}}%
  \addtocounter{footnote}{-1}%
  \endgroup
}
\renewcommand*{\multicitedelim}{\textcolor{seaborn_bg_grey_darker}{\addsemicolon}}
\setbeamerfont{footnote}{size=\scriptsize}
\renewcommand\footnoterule{\kern-3pt \color{seaborn_bg_grey_darker}\hrule width \textwidth height 0.4pt \color{black} \kern 2.6pt}

\DeclareSourcemap{
  \maps[datatype=bibtex,overwrite=False]{
   \map{
     \step[fieldsource=journal,
           match={Journal of Chemical Theory and Computation},
           replace={JCTC}]
     \step[fieldsource=journal,
           match={Reviews of Modern Physics},
           replace={Rev. Mod. Phys.}]
     \step[fieldsource=journal,
           match={Reports on Progress in Physics},
           replace={Rep. Prog. Phys.}]
     \step[fieldsource=journal,
           match={Physical Review Letters},
           replace={Phys. Rev. Lett.}]
     \step[fieldsource=journal,
           match={Physical Review},
           replace={Phys. Rev.}]
     \step[fieldsource=journal,
           match={B - Condensed Matter and Materials Physics},
           replace={B}]
     \step[fieldsource=journal,
           match={Journal of Chemical Physics},
           replace={J. Chem. Phys.}]
     \step[fieldsource=journal,
           match={Annual Review of Materials Research},
           replace={Annu. Rev. Mater. Res.}]
   }
  }
}

\renewbibmacro{in:}{}
\DeclareFieldFormat{pages}{\mkfirstpage{#1}}
\beamertemplatenavigationsymbolsempty
\bibliography{references.bib}
\setbeamertemplate{bibliography item}[text]
\renewbibmacro{in:}{}
\AtEveryBibitem{\clearfield{title}}
\AtEveryBibitem{\clearfield{month}}
\AtEveryBibitem{\clearfield{pages}}
\DeclareNameAlias{default}{given-family}

% \renewcommand*{\bibfont}{\tiny}
\usepackage{amssymb}
\usepackage{epsfig}
\usepackage{psfrag}
\usepackage{wrapfig}
\usepackage{graphicx}
\usepackage{color}
\usepackage[table]{xcolor}
\usepackage{amsmath}
\usepackage{multimedia}
\usepackage{subcaption}
%\usepackage{style}
\usepackage{verbatim}
\usepackage{multicol}
\usepackage[table]{xcolor}
\usepackage{tabularx}
\usepackage{cleveref}
% Tikz
\usepackage{tikz}
\usetikzlibrary{positioning,shapes,arrows,backgrounds,fit,calc,external,trees,tikzmark,fadings}
\tikzfading[name=fade bottom,top color=transparent!0, bottom color=transparent!100]
% \tikzexternalize[prefix=tikzfigures/]
\tikzstyle{dummy} = []
\tikzstyle{line} = [draw, thick, -latex']
\tikzstyle{headless_line} = [draw, thick, -]
\tikzstyle{default}    = [rectangle, text centered, rounded corners, text=black, font=\sffamily\footnotesize, align=center]
\tikzstyle{default_text}    = [rectangle, text width=10cm, text=black,anchor=north west, font=\sffamily]
\tikzstyle{boxwhite} = [default, fill=white, rounded corners=0.1cm]
\tikzstyle{cp}    = [default, fill=seaborn_blue, text=white, text width=2.8cm, minimum height=0.5cm]
\tikzstyle{pw}    = [cp, fill=seaborn_green]
\tikzstyle{wannier90}    = [cp, fill=seaborn_cyan]
\tikzstyle{bespoke}    = [cp, fill=seaborn_magenta]
\tikzstyle{observable}    = [cp, fill=seaborn_red]
\tikzset{
  -|-/.style={
    to path={
      (\tikztostart) -| ($(\tikztostart)!#1!(\tikztotarget)$) |- (\tikztotarget)
      \tikztonodes
    }
  },
  -|-/.default=0.5,
  |-|/.style={
    to path={
      (\tikztostart) |- ($(\tikztostart)!#1!(\tikztotarget)$) -| (\tikztotarget)
      \tikztonodes
    }
  },
  |-|/.default=0.5,
}

\newlength{\myyshift}
\setlength{\myyshift}{0.05cm}

\usepackage{lipsum}
\usetikzlibrary{calc}
\newlength{\myfigscale}
\setlength{\myfigscale}{0.3cm}
\usepackage{smartdiagram}
\usesmartdiagramlibrary{additions}
\usepackage{multicol}
\usepackage{helvet}
% \usepackage{sansmath}
% \sansmath
\usepackage{cancel} % for \cancel
\usepackage[normalem]{ulem} % for sout (strike out)
\usepackage{tcolorbox}
\tcbuselibrary{skins,hooks}
\tcbset{colframe=structure,fonttitle=\bfseries,beamer, clip upper, boxsep=0pt, sharp corners=all, no shadow, left skip=0pt, right skip=0pt, coltext=white}

% For electron orbital diagrams
\usepackage{tikzorbital}
% Changing defaults
\pgfkeys{tikzorbital/drawLevel/width = 0.666666}
\pgfkeys{tikzorbital/drawLevel/style = {line width = 1pt, color = black!80, line cap = round}}
\pgfkeys{tikzorbital/drawLevel/spinlength = 0.666666}
\pgfkeys{tikzorbital/drawLevel/spinstyle = {very thick, color = black!80, -stealth}}

\input{seaborn_colours.tex}

% For tikz diagrams with nodes appearing on each slide
\tikzset{
  invisible/.style={opacity=0},
  visible on/.style={alt={#1{}{invisible}}},
  alt/.code args={<#1>#2#3}{%
    \alt<#1>{\pgfkeysalso{#2}}{\pgfkeysalso{#3}} % \pgfkeysalso doesn't change the path
  },
}

\usepackage{array}
\usepackage{multirow}
% \newcolumntype{L}[1]{>{\raggedright\let\newline\\\arraybackslash\hspace{0pt}}m{#1}}
% \newcolumntype{C}[1]{>{\centering\let\newline\\\arraybackslash\hspace{0pt}}m{#1}}
% \newcolumntype{R}[1]{>{\raggedleft\let\newline\\\arraybackslash\hspace{0pt}}m{#1}}
\newcolumntype{L}{>{\raggedright\arraybackslash}X}
\newcolumntype{C}{>{\centering\arraybackslash}X}
\newcolumntype{R}{>{\raggedleft\arraybackslash}X}

% For checklist
%\usepackage{enumitem}
%\newlist{todolist}{itemize}{2}
%\setlist[todolist]{label=$\square$}
\usepackage{pifont}
\newcommand{\cmark}{\ding{51}}%
\newcommand{\xmark}{\ding{55}}%
\newcommand{\done}{\rlap{$\square$}{\raisebox{2pt}{\large\hspace{1pt}\cmark}}%
\hspace{-2.5pt}}
\newcommand{\wontfix}{\rlap{$\square$}{\large\hspace{1pt}\xmark}}

\newcommand{\bra}[1]{\langle #1|}
\newcommand{\braket}[2]{\langle #1|#2\rangle}
\newcommand{\braopket}[3]{\langle #1|#2|#3\rangle}
\newcommand{\ket}[1]{|#1\rangle}
\newcommand{\nline}{\nonumber \\}
\newcommand{\Trace}{\mathsf{Tr}}

\renewcommand{\ttdefault}{pcr} % enables bold fixed width font
\numberwithin{equation}{section}
% \usefonttheme{professionalfonts}
%\usefonttheme[stillsansseriflarge,stillsansserifsmall]{serif}
\usepackage{siunitx,booktabs}
% \AtBeginDocument{\sisetup{math-rm=\mathsf, text-rm=\sffamily}}
\AtBeginEnvironment{frame}{\setcounter{footnote}{0}}

\newlength{\myimscale}


% For code blocks in latex
% Taken from https://github.com/daveyarwood/gruvbox-pygments
% N.B.
%  - frame must have [fragile]
%  - use \begin{onlyenv} not \only
%  - after a lot of mucking around, I created gruvbox_plain as another style
%    that exclusively uses gruvbox's bg and fg with no syntax highlighting
%  - use [autogobble] to remove leading indentations

\usepackage{minted}
\usemintedstyle{gruvbox-dark}
\definecolor{gruvbox_dark_bg}{HTML}{282828}
\definecolor{gruvbox_fg}{HTML}{ebdbb2}
\definecolor{kgrey}{HTML}{2b2828}
\setminted[python]{bgcolor=gruvbox_dark_bg}
\setminted[json]{bgcolor=gruvbox_dark_bg}
\setminted[shell-session]{style=gruvbox_plain, bgcolor=gruvbox_dark_bg}

% \lstset{breaklines,breakatwhitespace,breakautoindent=false,showstringspaces=false}
% \lstset{keywordstyle=\color{purple}}
% \lstset{identifierstyle=\color{blue}}
% \lstset{basicstyle=\fontfamily{pcr}\fontsize{9pt}{9pt}\selectfont}
% %\lstset{numbers=left, numberstyle=\tiny, stepnumber=1, numbersep=5pt}
% \lstset{linewidth=4.9in,xleftmargin=10pt}

\setbeamercolor{frametitle}{bg=kgrey,fg=white}
\setbeamerfont{normal text}{family=helvet}
\setbeamerfont{local structure}{family=helvet}

\setbeamercolor*{author in head/foot}{bg=seaborn_blue}
\setbeamercolor*{logo in head/foot}{bg=seaborn_blue,fg=white}
\setbeamercolor*{title in head/foot}{bg=seaborn_blue,fg=kgrey}
\setbeamercolor*{date in head/foot}{bg=seaborn_blue,fg=white}
\setbeamercolor{title}{fg=kgrey}
\setbeamercolor{under headline}{bg=seaborn_red}
\setbeamercolor{footline}{bg=seaborn_blue}
\setbeamercolor{caption name}{fg=seaborn_blue}
\setbeamercolor{block title}{bg=kgrey,fg=white}
\setbeamercolor{block body}{bg=seaborn_bg_grey,fg=black}

% Footnote style and colour
% No line over footnote
\setbeamercolor{footnote}{fg=seaborn_bg_grey_darker}
\setbeamertemplate{enumerate items}[default]
\setbeamertemplate{blocks}[default]
\setbeamertemplate{itemize items}{\normalsize $\bullet$}
\setbeamercolor{description item}{fg=seaborn_blue}
\setbeamercolor{enumerate item}{fg=seaborn_blue}
\setbeamercolor{itemize item}{fg=seaborn_blue}
\setbeamercolor{itemize subitem}{fg=seaborn_blue}
\setbeamercolor{itemize subsubitem}{fg=seaborn_blue}
\setbeamercolor*{bibliography entry title}{fg=seaborn_bg_grey_darker}
\setbeamercolor*{bibliography entry author}{fg=seaborn_bg_grey_darker}
\setbeamercolor*{bibliography entry location}{fg=seaborn_bg_grey_darker}
\setbeamercolor*{bibliography entry note}{fg=seaborn_bg_grey_darker}
% and kill the abominable icon
\setbeamertemplate{bibliography item}[text]

\setbeamerfont*{title in head/foot}{size=\small}
\setbeamerfont*{date in head/foot}{size=\small}
\setbeamerfont*{institute}{size=\Large}

\setbeamertemplate{frametitle}
{
  \leavevmode%
  \vspace{-20pt}
  \begin{beamercolorbox}[wd=\paperwidth,ht=1cm]{frametitle}
   \hspace{0.115em}
   \vphantom{P/p} \bf \insertframetitle \vspace{0.2cm}
   \end{beamercolorbox}%
  %  \vskip-0.6cm%
  % \begin{beamercolorbox}[wd=\paperwidth,ht=0.5ex]{under headline}%
  %   \end{beamercolorbox}%
	
}

\newcommand{\insertframeinfo}{\insertframenumber/\inserttotalframenumber}
\newcommand{\backupbegin}{
   \newcounter{finalframe}
   \setcounter{finalframe}{\value{framenumber}}
   \renewcommand{\insertframeinfo}{}
}
\newcommand{\backupend}{
   \setcounter{framenumber}{\value{finalframe}}
}


\setbeamertemplate{frametitle}
{
  \vspace{-1pt}
  \begin{beamercolorbox}[wd=\paperwidth,ht=0.8cm]{frametitle}
   \hspace{0.05em}
   \begin{minipage}[c]{0.8\textwidth}
     \bf \insertframetitle

   \end{minipage}
   \hfill
   \begin{minipage}{0.15\textwidth}
   \begin{flushright}
   \scriptsize \textbf{Edward Linscott}
   
   {\raisebox{-0.15cm}{\includegraphics[height=0.45cm]{logos/psi_on_transparent.png}}
   \textbf{|}\hspace{0.1cm}
   \raisebox{-0.02cm}{\textbf{\insertframeinfo}}}%
   \vspace{-0.1cm}
   \end{flushright}
   \end{minipage}
   \vspace{0.125cm}
  \end{beamercolorbox}%
}

\setbeamertemplate{title page}
{
  \leavevmode%
  \vbox{%
  \vspace{-0.15\paperwidth}%
  \noindent\begin{tcolorbox}[enhanced,watermark graphics=photos/psi_in_mist_large.jpg, width=1.01\paperwidth, height=0.741\paperwidth, watermark zoom=1.0, grow to left by=0.05\paperwidth, frame hidden, coltext=kgrey]

  \vspace{0.26\paperwidth}

  \begin{minipage}{\textwidth}
  %  \begin{flushright}
  %  \includegraphics[height=0.05\textheight]{logos/logo_marvel_color_transparent.png}
  %  \hspace{0.1ex}
  %  \includegraphics[height=0.05\textheight]{logos/SNF_logo_standard_web_color_pos_e.png}
  %  % \hspace{0.01\textheight}
  %  % \includegraphics[height=0.05\textheight]{logos/black_cropped.eps}
  %  \hspace{0.1cm}\hbox{}
  % \end{flushright}

  \begin{center} 
  \LARGE

  \textbf{\inserttitle}

  \large
  \textbf{\insertsubtitle}
  \end{center}
  \end{minipage}
  \end{tcolorbox}

  \vspace{-3.7em}
  \begin{tcolorbox}[width=0.976\paperwidth, enhanced, colback=kgrey, grow to left by=0.035\paperwidth,]
  %  \begin{center}
  %  \footnotesize\bf\insertauthor\quad \raisebox{0.1ex}{|} \quad \insertshortinstitute\ \quad \raisebox{0.1ex}{|} \quad THEOS Group Meeting \quad \raisebox{0.1ex}{|} \quad \insertdate \quad \raisebox{0.1ex}{|} \quad \includegraphics[height=1.5ex]{logos/SNF_logo_standard_web_sw_neg_e.png} \ \ \includegraphics[height=1.5ex]{logos/logo_marvel_color_transparent_inverted.png}
  %  \end{center}
  \vspace{-0.7ex}
  \hfill \footnotesize\bf\insertauthor \hfill \raisebox{0.1ex}{|} \hfill \raisebox{-1.1ex}{\includegraphics[height=3.4ex]{logos/psi_on_transparent.png}} \hfill \raisebox{0.1ex}{|} \hfill THEOS Group Meeting \hfill \raisebox{0.1ex}{|} \hfill \insertdate \hfill \raisebox{0.1ex}{|} \hfill \includegraphics[height=1.5ex]{logos/SNF_logo_standard_web_sw_neg_e.png} \ \ \includegraphics[height=1.5ex]{logos/logo_marvel_color_transparent_inverted.png} \ \hfill
  %  \end{flushright}
  \end{tcolorbox}
  }


	
}
%\setbeamerfont{frametitle}{series=\bfseries}
\setbeamertemplate{footline}
{
}

% Title slide %%%%%%%%%%%%%%%%%%%%%%%%%%%%%%%%%%%%%%%%%%%%%%%%%%%%%%%%%%%%%%%%%%%%%%%%%%%%%%%%%%%
\author{Edward Linscott}
\institute{PSI}
\date{? January 2024}
\title{Towards black-box Koopmans band structures}
\subtitle{or: getting lost down a pseudopotential-generation rabbit hole}
\begin{document}

\begin{frame}{Beethoven's late string quartets}
    \centering

    String Quartet No. 14 - Adagio ma non troppo e molto espressivo

    \vspace{6pt}

    \includegraphics[height=0.6\paperheight]{photos/beethoven.jpg}

    {\tiny \href{https://www.youtube.com/watch?v=JE_crvhG3Co\&t=454s\&ab_channel=DavidSukonick\#t=4m27s}{click here for video}}

    \onslide<2->{``indecipherable, uncorrected horrors'' -- Spohr}

    \onslide<3->{``After this, what is left for us to write?'' -- Schubert}

\end{frame}

\frame{\titlepage}
% \frame{\titlepage}

\begin{frame}{\normalsize Koopmans functionals give accurate band structures}
   \begin{minipage}[c]{0.35\textwidth}
      \includegraphics[width=\textwidth]{figures/fig_nguyen_prx_bandgaps.png}
   \end{minipage}
   \hspace{1em}
   \begin{minipage}[c]{0.6\textwidth}

      \footnotesize
      Mean absolute error (eV) across prototypical semiconductors and insulators

      \vspace{1ex}
      \begin{tabular}{c S[table-format = 2.2] S[table-format = 2.2] >{\color{seaborn_red}\bfseries}S[table-format = 2.2] >{\color{seaborn_red}\bfseries}S[table-format = 2.2] S[table-format = 2.2]}
                          & {PBE} & {G\textsubscript{0}W\textsubscript{0}} & {KI} & {KIPZ} & {QSG$\tilde{\mathrm{W}}$} \\
         \midrule
         \midrule
         $E_\mathrm{gap}$ & 2.54  & 0.56                                   & 0.27 & 0.22   & 0.18                      \\
         %                                  & {MAPE (\%)} & 48.28 & 12.10      & 7.0           \\
         \midrule
         IP               & 1.09  & 0.39                                   & 0.19 & 0.21   & 0.49                      \\
         %                                  & {MAPE (\%)} & 15.58 & 5.71                                   & 2.99 & 3.14   & 7.41
      \end{tabular}
   \end{minipage}

   \blfootcite{Nguyen2018}
\end{frame}

\begin{frame}{\normalsize Koopmans functionals give accurate band structures}
   
\begin{table}[t]
   \centering
   \scriptsize
   \begin{tabular}{r@{ $\rightarrow$ } l *{3}{d{2.2}} >{\color{seaborn_red}}S[table-format = 2.2] >{\color{seaborn_red}}S[table-format = 2.2] d{2.2} @{$\pm$} d{1.2}}
      \hline
      \hline
      \multicolumn{2}{c}{ }
                                & \multicolumn{1}{c}{PBE}
                                & \multicolumn{1}{c}{G\textsubscript{0}W\textsubscript{0}\footnote{\cite{Shishkin2007} for $E_g$ and \cite{Hybertsen1986} for the transitions;}}
                                & \multicolumn{1}{c}{scG$\tilde{\mathrm{W}}$\footcite{Shishkin2007a}}
                                & \multicolumn{1}{c}{
                                 \textcolor{seaborn_red}{\bfseries KI@[PBE,MLWFs]}}
                                & \multicolumn{1}{c}{
                                 \textcolor{seaborn_red}{\bfseries KIPZ@PBE}}
                                & \multicolumn{2}{c}{exp\footcite{Madelung2004}}                                                                                                                                                                   \\
      \hline
      \multicolumn{2}{c}{$E_g$} &
      0.49 &  1.06 & 1.14 &  1.16 &   1.15 & \multicolumn{2}{c}{1.17}\\
      $\Gamma_{1v}$ & $\Gamma_{25'v}$ & 11.97 & 12.04 &      & 11.97 & 12.09 & 12.5 &  0.6\\
      $X_{1v}$ & $\Gamma_{25'v}$ &  7.82 &       &      &  7.82       &       & \multicolumn{2}{c}{7.75}\\
      $X_{4v}$ & $\Gamma_{25'v}$ &  2.85 &  2.99 &      &  2.85 & 2.86 & \multicolumn{2}{c}{2.90}\\
      $L_{2'v}$ & $\Gamma_{25'v}$ &  9.63 &  9.79 &      &  9.63 &  9.74 &  9.3 &  0.4\\
      $L_{1v}$ & $\Gamma_{25'v}$ &  6.98 &  7.18 &      &  6.98 &   7.04 &  6.8 &  0.2\\
      $L_{3'v}$ & $\Gamma_{25'v}$ &  1.19 &  1.27 &      &  1.19 &       &  1.2 &  0.2\\
      $\Gamma_{25'v}$ &  $\Gamma_{15c}$ &  2.48 &  3.29 &      &  3.17  &  3.20 & 3.35 & 0.01\\
      $\Gamma_{25'v}$ &  $\Gamma_{2'c}$ &  3.28 &  4.02 &      &  3.95  &  3.95 & 4.15 & 0.05\\
      $\Gamma_{25'v}$ &        $X_{1c}$ &  0.62 &  1.38 &      &  1.28  &  1.31 & \multicolumn{2}{c}{1.13} \\
      $\Gamma_{25'v}$ &        $L_{1c}$ &  1.45 &  2.21 &      &  2.12  &  2.13 & 2.04 & 0.06\\
      $\Gamma_{25'v}$ &        $L_{3c}$ &  3.24 &  4.18 &      &  3.91  &  3.94 &  3.9 &  0.1\\
      \hline
      \multicolumn{2}{c}{MSE} & 0.35 &  0.02 &      &  0.01 &   0.03\\
      \multicolumn{2}{c}{MAE} & 0.44 &  0.21 &      &  0.14 &   0.17\\
      \hline
      \hline
   \end{tabular}

   % \textsuperscript{\emph{a}} this work;
   % \textsuperscript{\emph{b}} Ref.~\citenum{Shishkin2007} for $E_g$ and Ref.~\citenum{Hybertsen1986} for the transitions;
   % \textsuperscript{\emph{c}} Ref.~\citenum{Shishkin2007a};
   % \textsuperscript{\emph{d}} Ref.~\citenum{DeGennaro2022};
   % \textsuperscript{\emph{e}} Ref.~\citenum{Madelung2004}
\end{table}
\end{frame}

% \begin{frame}{\normalsize Koopmans functionals give accurate band structures}
%    \begin{figure}[t]
%       \centering
%       \begin{subfigure}{0.25\textwidth}
%          \includegraphics[width=\columnwidth]{figures/ZnO_lda.png}
%       \end{subfigure}
%       \begin{subfigure}{0.25\textwidth}
%          \includegraphics[width=\columnwidth]{figures/ZnO_hse.png}
%       \end{subfigure}
%       \begin{subfigure}{0.25\textwidth}
%          \includegraphics[width=\columnwidth]{figures/ZnO_ki.png}
%       \end{subfigure}
%       \begin{subfigure}{\textwidth} %<-- changed width
%          \centering
%          %    \renewcommand\tabularxcolumn[1]{m{#1}}% <-- added
%          %    \renewcommand\arraystretch{1.3}
%          %    \setlength\tabcolsep{2pt}% <-- added
%          \begin{tabular}{c S[table-format = 2.2] S[table-format = 2.2] S[table-format = 2.2] S[table-format = 2.2] >{\color{seaborn_red}\bfseries}S[table-format = 2.2] S[table-format = 2.2]}
%             ZnO                                  & {LDA} & {HSE} & {GW$_0$} & {scG$\tilde{\rm W}$} & {KI} & {exp}       \\
%             \hline
%             $E_\mathrm{gap}$ (eV)                & 0.79  & 2.79  & 3.0      & 3.2                  & 3.62 & 3.60        \\
%             $\langle \varepsilon_d \rangle$ (eV) & -5.1  & -6.1  & -6.4     & -6.7                 & -6.9 & {-7.5/-8.0} \\
%          \end{tabular}
%          %        \caption{table}
%       \end{subfigure}
%       % \caption{Band structure of ZnO calculated at different level of theory:
%       %    LDA (left panel), HSE (middle panel) and KI (right panel). Shaded areas
%       %    highlight valence (light blue) and conduction (light red) manifolds. The
%       %    experimental values for the band gap and for the energy position of
%       %    Zn $d$-states are represented by the dashed green line and by the dashed
%       %    red line, respectively.
%       %    Table: Band gap and position of Zn $d$ states with respect to the top of the valence band at different level of theory compared to experimental and GW results from Ref.~\onlinecite{shishkin_accurate_2007}.}
%    \end{figure}
%    \blfootcite{Colonna2022}
% \end{frame}

% \begin{frame}{Koopmans functionals: results for molecules}
%    \small
%    Ionisation potentials $ = E(N-1) - E(N) \stackrel{?}{=} -\varepsilon_{HO}$ of 100 molecules (the GW100 set) cf. CCSD(T)
%    \begin{center}
%       \includegraphics[height=0.2\textwidth]{figures/colonna_2019_gw100_ip}
%       % \onslide<2->{\includegraphics[height=0.23\textwidth]{figures/colonna_2019_gw100_deeper}}
%    \end{center}
% 
%    \vspace{-3ex}
%    Ultraviolet photoemission spectra
%    \begin{center}
%       \begin{tikzpicture}
%          \node [inner sep=0pt](fig) at (0,0) {\includegraphics[height=0.35\textheight]{figures/fig_nguyen_prl_spectra.png}};
%          \draw [very thick, color=seaborn_red] (-5.35,-0.07) rectangle (5.4,1.6);
%       \end{tikzpicture}
%    \end{center}
%    \vspace{-2ex}
% 
%    \blfootcite{Colonna2018,Nguyen2015}
% \end{frame}

% \begin{frame}{Koopmans functionals: results for molecules}
%    Electron affinities $ = E(N) - E(N+1) \stackrel{?}{=} -\varepsilon_{LU}$ of molecules cf. CCSD(T)/exp
%    \vspace{2ex}
% 
%    \small
%    \begin{center}
%       For 15 of the GW100 molecules with bound LUMOs
% 
%       \includegraphics[height=0.5\textheight]{figures/fig_gw100_ea_mae_mse.pdf}
% 
%       \textcolor{seaborn_bg_grey_darker}{\footnotesize Linscott et al. (in prep)}
%    \end{center}
% \end{frame}

% \begin{frame}{Koopmans functionals: results for toy systems}
%    For Hooke's atom (two electrons in a harmonic confining potential with Coulombic repulsion)
% 
%    \begin{figure}[t]
%       \begin{subfigure}{0.4\textwidth}
%          \includegraphics[width=\columnwidth]{figures/schubert_vxc.jpeg}
%       \end{subfigure}
%       \begin{subfigure}{0.4\textwidth}
%          \onslide<2->{
%          \includegraphics[width=\columnwidth]{figures/schubert_vxc_integrated.jpeg}
%          }
%       \end{subfigure}
%    \end{figure}
% 
%    \blfootcite{Schubert2023}
% \end{frame}

\begin{frame}{Features of Koopmans functionals}
   \begin{align*}
      E_\mathsf{Koopmans}[\rho,{\{f_i\}}, {\{\alpha_i\}}]
      = {E_{DFT}[\rho]}
      + \sum_i
      {\alpha_i}
      \Biggl(
      -
         {\int^{f_i}_{0} \varepsilon_i(f) df}
      +
         {f_i {\eta_i}}
      \Biggr)
   \end{align*}
   General features:
   \begin{itemize}[<+(1)->]
      \item a correction to DFT that enforces a generalized piecewise linearity condition
      \item is orbital-density-dependent
      \item relies on localization
      \item requires the ab initio calculation of screening parameters
   \end{itemize}
   
   \onslide<6->{In order to evaluate this functional, one must...}
   \begin{itemize}[<+(1)->]
      \item initialize a set of variational orbitals
      \item calculate the screening parameters $\{\alpha_i\}$
      \item construct and diagonalize the Hamiltonian
   \end{itemize}
\end{frame}

\begin{frame}{Workflows}
   \onslide<2->{
      (a) finite difference calculations using a supercell

      \vspace{-2ex}
      \adjustbox{width=\textwidth}{\input{supercell_workflow.tex}\end{tikzpicture}}
   }

   \vspace{-1.5ex}
   \onslide<3->{
      (b) DFPT using a primitive cell

      \vspace{-2ex}
      \adjustbox{width=0.655\textwidth}{\input{primitive_workflow.tex}}
   }

   \vspace{2ex}
   \onslide<4->{All implemented in \raisebox{-0.65ex}{\includegraphics[height=2.5ex]{figures/koopmans_grey_on_transparent.png}}}

   \blfootcite{Linscott2023}

\end{frame}

\begin{frame}{Workflows}

   What still stands in our way? Take the example of silicon:

   \vspace{-1ex}
   \inputminted[fontsize=\scriptsize]{json}{scripts/si.json}

\end{frame}

\begin{frame}{Automating Wannierization}

   One very manual step: Wannierization. Can we automate this?

   \begin{figure}[t]
      \begin{subfigure}{0.2\textwidth}
         \onslide<2->{
         \includegraphics[height=1.5\columnwidth]{figures/proj_disentanglement_fig1b.png}
         \vspace{-0.01\paperheight}
         }
      \end{subfigure}
      \begin{subfigure}{0.2\textwidth}
         \onslide<3->{
         \includegraphics[height=1.5\columnwidth]{figures/proj_disentanglement_fig1a.png}
         }
      \end{subfigure}
      \hspace{0.025\textwidth}
      % \begin{subfigure}{0.225\textwidth}
      %    \onslide<3->{
      %    \includegraphics[height=1.5\columnwidth]{figures/proj_disentanglement_fig1d.png}
      %    }
      % \end{subfigure}
      \begin{subfigure}{0.2\textwidth}
         \onslide<4->{
         \includegraphics[height=1.5\columnwidth]{figures/proj_disentanglement_fig1f.png}
         }
      \end{subfigure}
   \end{figure}

   \blfootcite{Qiao2023}

\end{frame}

\begin{frame}{Automating Wannierization}
   We separate target manifolds via parallel transport to obtain separate occupied and empty manifolds

      \begin{figure}
         \includegraphics[width=0.3\columnwidth]{figures/target_manifolds_fig1b.png}
      \end{figure}

   \blfootcite{Qiao2023a}

\end{frame}

\begin{frame}{Automating Wannierization}

   Pseudoatomic orbitals play a dual purpose
   \begin{itemize}
      \item we use them to calculate projectability \onslide<2->{$\rightarrow$ we rely on the PAOs having high overlap with the atomic-like bands}
      \item they serve as our initial guesses for the Wannier functions \onslide<3->{$\rightarrow$ the number of Wannier functions is limited to the number of pseudoatomic orbitals}
   \end{itemize}

   \onslide<4->{e.g. LiF}

   \onslide<5->{Li 1s\textsuperscript{2} 2s\textsuperscript{1}} \onslide<6->{$\rightarrow$ 2 PAOs (1s, 2s)}

   \onslide<7->{F \textcolor{seaborn_bg_grey_darker}{[1s\textsuperscript{2}]} 2s\textsuperscript{2} 2p\textsuperscript{5}} \onslide<8->{$\rightarrow$ 4 PAOs (1s, 2s, 2p\textsubscript{x}, 2p\textsubscript{y}, 2p\textsubscript{z})}

   \onslide<9->{6 Wannier functions for a system with 10 electrons} \onslide<10->{= 5 occupied bands + only 1 unoccupied band!}

   \onslide<11->{If we want more Wannier functions, we're gonna need \sout{a bigger boat} more PAOs...}

\end{frame}

\begin{frame}{Automating Wannierization}
    Existing strategy: use the PAOs provided by OpenMX

    \includegraphics[width=0.8\textwidth]{figures/Li8.0.pao_2_2_1_0.png}

    \includegraphics[width=0.8\textwidth]{figures/Li8.0.pao_10_0_0_0.png}

\end{frame}

\begin{frame}{\footnotesize Maybe the real treasure is the python packages we made along the way}

    \includegraphics[width=0.8\textwidth]{figures/upf_tools.png}

\end{frame}

\begin{frame}{\footnotesize Maybe the real treasure is the python packages we made along the way}

    \begin{itemize}
        \item Most of the infrastructure was from the \texttt{cookiecutter} I used (see my July 2022 GM)
        \item Publish your code! \texttt{upf-tools} was discovered by Marnik and is now used in \texttt{AiiDA}
        \item Also have unmerged tools for \texttt{oncvpsp} input and output files as well as Junfeng's custom \texttt{.dat} projector files (perhaps \texttt{oncvpsp-tools} Need to decide on the right home for these...
    \end{itemize}
    
\end{frame}

\begin{frame}{Take home messages}

   % \includegraphics[height=0.2\paperheight]{figures/colonna_2019_gw100_ip.jpeg}
   % \hfill
   \includegraphics[height=0.25\paperheight]{figures/fig_nguyen_prx_bandgaps.png}
   \hfill
   \includegraphics[height=0.275\paperheight]{figures/proj_disentanglement_fig1a.png}
   \hfill
   \adjustbox{height=0.275\paperheight}{\input{supercell_workflow.tex}\end{tikzpicture}}

   \begin{itemize}
      \item Koopmans functionals yield band structures with comparable accuracy to state-of-the-art GW
      \item the release of \texttt{koopmans} means non-experts can now use Koopmans functionals in their own research
      \item work is ongoing to automate the Wannierization bottleneck
   \end{itemize}

\end{frame}

\begingroup
\setbeamertemplate{footline}{}
\begin{frame}{Acknowledgements}
   \begin{center}
      \footnotesize
      \begin{tabularx}{\textwidth}{CCCC}
         \includegraphics[height = 0.3\paperheight]{figures/nicola_marzari.jpg}     &
         \includegraphics[height = 0.3\paperheight]{figures/nicola_colonna2.png}    &
         \includegraphics[height = 0.3\paperheight]{photos/riccardo_degennaro.jpeg} \\
         % \includegraphics[height = 0.2\paperheight]{figures/daniel_cole.jpeg}       &
         % \includegraphics[height = 0.2\paperheight]{figures/mike_payne.jpeg}        &
         % \includegraphics[height = 0.2\paperheight]{figures/david_oregan.jpg}         \\
         Nicola Marzari                                                             &
         Nicola Colonna                                                             &
         Riccardo De~Gennaro                                                        &
         Junfeng Qiao
      \end{tabularx}
   \end{center}

   \vspace{2ex}

   \begin{center}
      \includegraphics[height = 0.15\paperheight]{logos/SNF_logo_standard_print_color_pos_e.eps}
      \hspace{3em}
      \includegraphics[height = 0.15\paperheight]{figures/marvel_trimmed.png}
   \end{center}

   \vspace{1ex}

   \begin{center}
      slides available at \includegraphics[height=\fontcharht\font`\B]{logos/github-favicon.png} github/elinscott-talks and on the THEOS wiki
   \end{center}

   % \begin{multicols}{2}
   %    \tiny
   %    \printbibliography
   %    \normalsize
   % \end{multicols}
   \vspace{2ex}
   \scriptsize

   \setbeamercolor*{bibliography entry title}{fg=black}
   \setbeamercolor*{bibliography entry author}{fg=black}
   \setbeamercolor*{bibliography entry location}{fg=black}
   \setbeamercolor*{bibliography entry note}{fg=black}

   \vspace{2ex}
   \scriptsize
\end{frame}
\endgroup
% \backupbegin
% \begin{frame}{}
% 
%     \begin{center}
%         \huge SPARE SLIDES
%     \end{center}
% 
% \end{frame}
% 
% \begin{frame}{Spare slide}
% \end{frame}
% 
% 
% \backupend
\end{document}
